
\chapter*{Declaration and Impact statement}
\addcontentsline{toc}{chapter}{Declaration and Impact statement}


\begin{flushright}{    
All models are wrong, but some are useful \\} \medskip
--- George Box, statistician
\end{flushright}

\bigskip



This thesis is an account of research undertaken between February 2021 and  October 2021 at Research School of Biology, Joint College of Sciences, The Australian National University, Canberra, Australia.

Except where acknowledged, the material presented in this thesis is, to the best of my knowledge, original and has not been submitted in whole or part for a degree in any university.

At the beginning of the year, my Honours cohort and I all had a model in our head that we would be able to go on campus everyday. Towards the end of the year, however, everyone had to move to work-from-home mode due to the lockdown. As per requested by the Honours convenor Spencer Whitney, I get to complain about how I was affected in this paragraph. Regarding the lockdown, I am luckier than many students because my project is purely computational. At the time of moving to work from home, my supervisors and I discovered that we had to reapply for access to the restricted proportion of the data. I am not sure whether it was the consequence of the lockdown, but the application was not completed by the time I had to submit my thesis. Therefore, we decided to do more analyses on the data that I had instead of waiting.

\vspace{20mm}  % vertical space

\large
\hspace{12cm} Phuong Le\par
\hspace{12cm} \today

\normalsize
\chapter*{Acknowledgements}
\addcontentsline{toc}{chapter}{Acknowledgements}

First, I would like to thank my supervisor Gavin Huttley. Gavin started mentoring me two years ago when I took his course, Bioinformatics, with hardly any programming background. I then did a short project with him the following semester. Finally, I came back to undertake Honours with him this year. Over such a long period of time, my view remains the same, that I am privileged to be his student. He is one of the rare true bioinformaticians who have a profound understanding of all three domains: biology, computing and statistics. He genuinely cares about students' development, not just the outcome of our work. I always feel extremely assured when I talk with him, when he tells me that he has ``complete confidence in me'', to ``do good science'' and to ``do what we believe is right''. His lab is a family where Katherine is apparently the bad twin and I am the evil twin. I cannot express how grateful I am for being instructed by Gavin. I hope my thesis and everything I do later on in life will make him proud.

Second, I would like to thank my co-supervisor, Cheng-Soon Ong. I have a tremendous respect for his expertise in Machine learning and his superpower of communicating with biologists. While being a busy director of a machine learning platform at CSIRO, Cheng always makes time for me whenever I ping him on Slack. He connects me with his brilliant students and employees at Data61 so that I have even more people to hassle. I wish I could quote everything he said, but let me just quote one: ``Be kind''.

I would also like to thank the three powerful individuals who control my fate, Aude Fahrer, Eric Stone and Sasha Mikheyev. I would like to acknowledge Eric, whose statistical mindset I find particularly eye-opening. With his experience, Eric's comments and advice have always been spot on. He has also referred me to his employees at the BDSI, whose expertise was in line with what I was doing. I cannot ask for a more helpful examiner. In addition, I would like to thank Aude, who I knew from her first year cancer lectures and her immunology course. Her input on the biological motivation for my project was invaluable. Communicating with her gives me the perfect chance to practice intellectual conversations with people outside my area.

Throughout the project, I had the opportunity to receive technical support from so many people. I wanted to thank anh\footnote{anh: Vietnamese for big brother} Mai Huy Hoang for patiently helping me with the ``computer stuff'' ever since we were high school students until now when he's finishing his degree at the prestigious Peking University. I wanted to thank my ge\footnote{ge: Chinese for big brother} Zixian Cai for his advice on everything, because conveniently for me, he is both a brilliant computer scientist and a senior international student of the same degree as me. I was also lucky to get help from Cheng's PhD student, Mengyan Zhang, who understands how to implement machine learning algorithms very well. She is kind enough to volunteer to give me advice that always points out both the strengths and weaknesses of my crazy ideas... I wanted to thank anh Hoang Danh Tai at the BDSI for sharing his machine learning experience and James Nichols for discussing Wasserstein distance with me. I wanted to thank Aaron Chuah at the JCSMR for sharing his wide knowledge of genomic databases. 

Thank you to lovely people who patiently read and edited my long thesis: 

In terms of financial support, I would like to thank the ANU for awarding me the annual Terrell International Undergraduate Scholarships. On top of that, I would like to thank my past and present employers for letting me work for them to partly support myself when I moved to Australia, they are Susan Howitt, Mark Ellison, Quynh Nguyen, Johannes Zoller, Thuan Truong and Sao Berger. Susan Howitt, in particular, was another mentor of mine. She taught me during her genetics course, gave me honest advice in terms of career development and course selection. She was my referee when I transferred into the PhB. This year, I worked for her as a demonstrator and I can say that she was extremely caring as an employer. 

This year was ridiculously intense for me in many ways - everyone who witnessed it would agree and I thank friends who were by my side to help me through it. Thank you to Katherine Caley, my lab mate. Throughout the year, Kath knows virtually everything I went through, she is there for me and she helps as much as she can. She enjoys our complaint sessions, she exchanges code with me and co-authors some of my emails. Thanks to my roommate Chelsea Crew and landlords Brett and Barbara Still for their amazing kindness. Thanks to em\footnote{Vietnamese for little sister/brother} Han Do and Randolph Leong for being great friends and exchanging food with me. Thanks to my old roommate Hanh Vo for supporting me with useful materials even though she is now based in Tasmania (too bad you couldn't be my source of distraction as I was for you last year). Thanks to my aunt Le Kim Thuy, who has constant fear that I will starve myself to death and makes me heaps of delicious food. Thanks to my best friend Nguyen Thi Ha Anh, who always supports my superstition by being professionally supersitious herself. Thanks to Soraya Zwahlen, Hang Dang, Minh Bui, Thinh Ngo, Emily Tan, Sharwel Lei, Anya Zhao, Nigel Heng. And Maksim Lisau. 

Another important person who got me through Honours was Spencer Whitney, who convened my Honours cohort. He is extremely experienced with students and he is always ready to help. He has a questionable sense of humour - he likes jokes that target my supervisor Gavin Huttley. I do too. 

Last but not least, I wanted to thank my family back in Vietnam, my mum Nguyen Thanh Thu, my dad Le Minh Quang and my sister Le Khanh Huong. My parents have been supporting me unconditionally. They are my main financial support, without which my study at ANU would never be possible. But most importantly, they are my biggest emotional support that I can always rely on. When I am tired, I think of home. I guess there is no need to say anything further. 