\begin{figure}[htbp]
    \begin{subfigure}{.5\textwidth}
\includegraphics[scale=0.7]{graphics/spectra_Kidney-RCC_Skin-Melanoma.pdf}
    \caption{Kidney-RCC \textit{v.s.} Skin-Melanoma}
    \label{fig:spectra_kidney_skin}
    \end{subfigure}
    ~
    \begin{subfigure}{.5\textwidth}
    \includegraphics[scale=0.7]{graphics/spectra_Kidney-RCC_Liver-HCC.pdf}
    \caption{Kidney-RCC \textit{v.s.} Liver-HCC}
    \label{fig:spectra_kidney_liver}
    \end{subfigure} \\
    \vspace{0.5cm}
    \caption{\textbf{Base substitutions are promising in discriminating cancers according to $RE$ between cancer pairs} Similar to Figure \ref{fig:spectra}, for each panel, each row was derived from a GLM. The x-axis is the wildtype base; the y-axis is the product of the substitution. The heights of the letters are $RE$'s. An up-orientation indicates an excess while a down-orientation indicates a deficit of the mutation when comparing the Kidney-RCC to (a) Skin-Melanoma, (b) Liver-HCC.}
    \label{fig:paired_spectra}
\end{figure}