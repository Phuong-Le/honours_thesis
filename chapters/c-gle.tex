\chapter{Genomic Location Effect}\label{gle}

The tendency for mutations to occur in closed \gls{chromatin} regions has been reported both in cancer and other mutagenesis processes \citep{Polak2015,Prendergast2007ChromatinGenome}. This is hypothetically because closed chromatin regions, despite being less exposed to mutagens, are harder for repair systems to reach \citep{Prendergast2007ChromatinGenome,Teng1997ExcisionSequences, Morse2002PhotoreactivationCerevisiae}. Based on the premise that different cell-types harbour different chromatin structures \citep{Kundaje2015IntegrativeEpigenomes}, it is reasonable to expect that when developed into tumours, mutations are allocated differently between different cell-types. This chapter shows further evidence, through a G-test of independence and the \gls{or} statistic, that in cancers, mutations do tend to occur in closed rather than open chromatin regions, with the exception of breast adenocarcinoma and rectal cell carcinoma. The degree at which this phenomenon occurs varies across cancer types. In addition, the chapter also uses the \gls{bootstrap} to confirm that \gls{gle} alone, without the chromatin status input, are significantly different in different cancers. 

\section{Mutation location is influenced by chromatin status}
My analyses of GLE were in accordance with the observation that mutations tend to locate in closed chromatin regions. Figure \ref{fig:mutation_density} shows the distribution of mutations on chromosome 12 for four cancers, the rest can be found in Figure \ref{fig:apdx_mutation_density} of the appendix. The green DHS bars near the bottom of the plots are hypersensitive regions, so chromatin is open at regions with dense DHS bars. The choice to display chromosome 12 was arbitrary. By visualisation alone, it can already be seen that mutation density tends to peak at regions with less dense DHS bars, indicating a bias towards closed chromatin regions. This pattern is particularly strong in Skin-Melanoma and Liver-HCC, and less obvious in Kidney-RCC. It is also worth noting that the chromatin structures are different for different cancers. Looking at the density by itself, we could see a diversity in how mutations are distributed, supporting the potential of GLE in discriminating cancers. Unsurprisingly, there are certain conserved patterns for the cancers, which hypothetically correspond to important characteristics of human cells. For example, the level of mutations tend to drop around position 5$\times 10^7$, at which chromatin is closed for all cell types.  

\begin{figure}
    \begin{subfigure}{.5\textwidth}
    
    \includegraphics[width=\linewidth,height=0.7\textwidth]{graphics/mutdistribution_melanoma.png}
    \caption{Skin-Melanoma}
    \label{fig:density_skin}
    \end{subfigure}
    ~
    \begin{subfigure}{.5\textwidth}
    
    \includegraphics[width=\linewidth,height=0.7\textwidth]{graphics/mutdistribution_kidney.png}
    \caption{Kidney-RCC}
    \label{fig:density_kidney}
    \end{subfigure} \\
    
    \begin{subfigure}{.5\textwidth}
    
    \includegraphics[width=\linewidth,height=0.7\textwidth]{graphics/mutdistribution_Liver-HCC.png}
    \caption{Liver-HCC}
    \label{fig:density_liver}
    \end{subfigure}
    ~
    \begin{subfigure}{.5\textwidth}
    
    \includegraphics[width=\linewidth,height=0.7\textwidth]{graphics/mutdistribution_Panc-AdenoCA.png}
    \caption{Panc-AdenoCA}
    \label{fig:density_panc_adenoca}
    \end{subfigure} \\
    
    \caption{\textbf{Density}. The x-axis indicates the original base of a SBS, the y-axis indicates the base that it is mutated to. The height of a letter at a SBS is the $RE$ value for that SBS. If the letter is upside down, there is a decrease of abundance when moving from the first to the second cancer and vice versa.}
    \label{fig:mutation_density}
\end{figure}

\subsection{Open and closed chromatin regions have significantly different mutation rates}
Having visualised the tendency of mutation location, I performed a G-test of independence to see whether a formal hypothesis test can confirm that there is a difference in how mutations are distributed between open and closed chromatin regions. The p-values estimated from this were adjusted by Bonferroni multiple test correction and shown in Table \ref{tab:g-test}; the raw inputs can be found in Appendix \ref{apdx:g-test}. To begin with, the size of the regions identified as open chromatin is considerably tiny compared to that of closed chromatin regions. Keeping that in mind, we can see that most cancers had significantly different mutation rates between open and closed chromatin regions, except CNS-PiloAstro and Panc-Endocrine. Note that these two cancers had small to modest numbers of mutations. While no direct correlation between p-values and number of mutations could be detected, no cancers with less than 1 million mutations gave a p-value $>10^{-100}$. Our conclusion remains that mutation location is not random between open and closed chromatin regions, but this implies the impact of mutation load on the power of the test. 

% latex table generated in R 4.1.0 by xtable 1.8-4 package
% Tue Oct 19 08:53:20 2021
\begin{table}[h]
\centering
\caption{\textbf{The chance of mutations occurring are significantly different between closed and open regions for most cancers.} The table presents the estimated p-values, adjusted by Bonferroni multiple test correction, from the independent G-test. Taking the significance threshold at 0.05, all cancers have different rates of mutations between open and closed chromatin regions, with the exception of CNS-PiloAstro and Panc-Endocrine.}
\label{tab:g-test}
\begin{tabular}{lrr}
  \toprule
 \textbf{Disease} & \textbf{$\hat{p}$-value} & \textbf{Number of mutations} \\ 
  \hline
 Bone-Osteosarc & 5.66 $\times 10^{-35}$ & 166845 \\ 
 Breast-AdenoCa & 1.33 $\times 10^{-10}$ & 713855 \\ 
 CNS-Medullo & 2.46 $\times 10^{-34}$ & 209997 \\ 
 CNS-PiloAstro & 1.00 & 22020 \\ 
 Kidney-RCC & 2.60 $\times 10^{-06}$ & 531886 \\ 
 Liver-HCC & $<10^{-100}$ & 3321521 \\ 
 Lymph-BNHL & $<10^{-100}$ & 1124881 \\ 
 Lymph-CLL & $<10^{-100}$ & 226242 \\ 
 Panc-AdenoCA & $<10^{-100}$ & 1675781 \\ 
 Panc-Endocrine & 8.82 $\times 10^{-02}$ & 258564 \\ 
 Prost-AdenoCA & 5.49 $\times 10^{-90}$ & 1000496 \\ 
 Skin-Melanoma & $<10^{-100}$ & 7770980 \\ 
   \bottomrule
\end{tabular}
\end{table}

\subsection{Mutations are biased towards closed regions, with variation}
Complementary to the G-tests, which suggests that the difference in mutation location were statistically significance between closed and open chromatin regions, I computed the odds ratio ($OR$), which measures the direction of this difference. From equation \ref{eq:or}, $OR$ compares the ratios of mutated over non-mutated positions between closed and open regions. An $OR<1$ indicates a bias towards closed regions, and an $OR>1$ indicates a bias towards open region. For each cancer, I estimated the possible range of $OR$'s using jackknife. The results are shown in Figure \ref{fig:or_jackknifed}.

\begin{figure}[h!]
    \centering
    \includegraphics[scale=0.75]{graphics/jackknife_OR.pdf}
    \caption{\textbf{Mutations tended to occur in closed chromatin regions according to the odds ratio ($OR$) statistic}. $OR>1$ indicates a bias towards towards closed regions, and $OR<1$ indicates the opposite. Error bars are the standard errors of the jackknifed sample. The green circles are the means of the jackknifed pseudo-values. The purple stars are the observed $OR$.}
    \label{fig:or_jackknifed}
\end{figure}


Overall, mutations preferred to locate in closed chromatin regions, especially for Skin-Melanoma ($OR=2$). However, this bias varies for different cancers. In addition, there are three other intriguing features. First, Breast-AdenoCa and Kidney-RCC had $OR<1$. Their p-values from the G-test showed a significant difference between closed and open regions, so this preference for open regions is less likely to come from noise. On the density plot (Figures \ref{fig:mutation_density} and \ref{fig:apdx_mutation_density}), mutations still peak at closed chromatin regions, but less clear than for example Skin-Melanoma. Second, the departure of the observed $OR$ from the mean pseudo-values in Lymph-BNHL and Lymph-CLL suggests some abnormality in these cases. This abnormality could be due to the large variance of GLE in donors with these cancers. However, it could also come from the uncertainty in identification of the original cells (B cells fro both cancers, discussed in \ref{}). Third, CNS-PiloAstro has a widest estimated range of $OR$. Again, this might be because its small number of mutations lowered the signal to noise ratio compared to other cancers, which is consistent with the G-test. Regarding reliability, empirically, $OR$ seems more robust to the number of mutations than G-test. Mathematically, it accounts for the unbalance in the size of open \textit{v.s.} closed chromatin regions. However, we need to be vigilant about the existence of this the unbalance.

\subsection{$OR$ is not influential but not discriminative}

In this subsection, I evaluated the appropriateness of $OR$ and its potential in discriminating cancers. This was motivated by the unbalance observed between the closed and open chromatin regions. The evaluation was done by calculating the $OR$'s when sorting a cancer's mutation data based on other cancers' DHS data. 

The impact of unbalanced DHS was assessed in Figure \ref{fig:mixed_or_violin}. If $OR$ was sensitive to the unbalance, then each violin should have had a distinctive value. A good example to illustrate this involves Skin-Melanoma and Kidney-RCC, whose ratios of closed over open chromatin regions were 65 and 115 in their original cells, respectively (Table \ref{fig:tab_g-test_contingency}). If $OR$ was sensitive, the it should have always been higher in Kidney-RCC than Skin-Melanoma, no matter what mutation data was used. This suggests that the effect of unbalanced DHS data was mild.

However, it is curious that $OR$ was not typically the highest when DHS data when the true cancer was used. This was further reinforced in Figure \ref{fig:mixed_or_heatmap}, where each column, representing DHS data, was coloured with respect to the rank of $OR$. Mutation data for Skin-Melanoma almost always produced the highest $OR$, irrespective of the cancer types used for DHS data. Accordingly, chromatin structure, in the form of $OR$ is indicative of mutation location, but it is unlikely to be the main determinant of whether GLE differs for different cancers. One possible explanation is the similarities in the DHS data of the original cells.

\begin{figure}[htbp]
    \begin{subfigure}{.5\textwidth}
    \includegraphics[width=\linewidth,height=0.7\textwidth]{graphics/o}
    \caption{Skin-Melanoma}
    \label{fig:density_skin}
    \end{subfigure}
    ~
    \begin{subfigure}{.5\textwidth}
    
    \includegraphics[width=\linewidth,height=0.7\textwidth]{graphics/mutdistribution_Kidney-RCC.png}
    \caption{Kidney-RCC}
    \label{fig:density_kidney}
    \end{subfigure} \\
    \vspace{0.5cm}

    \caption{\textbf{Mutations tend to be found in closed chromatin regions.} Different cancers differ in the distribution of mutations across the genome. Here chromosome 12 is shown. (a) Skin-Melanoma (b) Kidney-RCC (c) Liver-HCC (d) Panc-AdenoCA. The green bar below the x-axis indicates hypersensitivity regions, the gap indicates closed chromatin regions of the original cell types. The vertical dotted line indicates the position of the centromere.}
    \label{fig:mutation_density}
\end{figure}

\newpage
\section{GLE is significantly different for different cancers}
Previously, we observed different patterns of GLE for different cancers by visualisation (Figure \ref{fig:mutation_density}). In this section, I investigated whether the difference is statistically significant and what data representations are optimal for GLE. For each pair of cancers, I used a bootstrap hypothesis test for whether their GLE are significantly different. I trialled two representations (bin and smoothing) and two distance measures (Euclidean and Wasserstein) to see which representation/measure could best discriminate cancers. I reported the raw p-values rather than recruiting multiple test correction because if two reasons. First, the true purpose of this section was to detect signals in the data, so it was not particularly meaningful to set a rigid significance threshold. Second, each p-value was estimated from 1000 simulations of a cancer pair; consequently, the p-values obtained were not drawn from the same distribution. From Table \ref{tab:gle_bootstrap}, GLE was generally very different for all cancer pairs. The smoothing representation were more likely to output more significant p-values, with only two pairs at p-values $>0.001$ for Wasserstein and no pairs for Euclidean distance. The most common cancer with p-value $>0.001$ was CNS-PiloAstro, which might be due to its small sample size, as in the case with the G-test and the $OR$. 

\begin{table}[!htb]
    \caption{\textbf{Estimated p-values from bootstrap hypothesis tests involving 1000 simulations for (a) Bin/Euclidean, (b) Bin/Wasserstein, (c) Smooth/Euclidean, (d) Smooth/Wasserstein} No multiple test correction was applied. All estimated p-values were $<0.001$ unless otherwise specified.}
    \label{tab:gle_bootstrap}
    % row 1
    \begin{subtable}[!h]{.5\textwidth}
        \centering
        \begin{tabular}{ p{3cm}p{2.7cm}c }
        & \textbf{CNS-PiloAstro} & \textbf{others} \\
        \textbf{CNS-Medullo} & 0.121 & - \\
        \textbf{Kidney-RCC} & 0.024 & - \\
        \textbf{Liver-HCC} & 0.121 & - \\
        \textbf{Panc-Endocrine} & 0.069 & - \\
        \textbf{Prost-AdenoCA} & 0.009 & - \\
        \textbf{all others} & - & - \\
        \end{tabular}
        \vspace{0.2cm}
    \subcaption{Bin/Euclidean}
    \end{subtable} 
    \quad % for side by side tables
    \begin{subtable}[!h]{.5\textwidth}
        \centering
        \begin{tabular}{ p{2.9cm}p{2.7cm}c }
        & \textbf{CNS-PiloAstro} & \textbf{others} \\
        \textbf{Bone-Osteosarc} & 0.034 & - \\
        \textbf{Panc-AdenoCA} & 0.024 & - \\
        \textbf{Skin-Melanoma} & 0.017 & - \\
         &  &  \\
         &  &  \\
        \textbf{all others} & - & - \\
        \end{tabular}
        \vspace{0.2cm}
    \subcaption{Bin/Wasserstein}
    \end{subtable}  
    % row 2
    \begin{subtable}[!h]{.5\textwidth}
        \centering
        \begin{tabular}{ p{3cm}p{2.7cm}c }
        &  & \textbf{All} \\
        &  &  \\
        &  &  \\
        \textbf{All} & - & - \\
        \end{tabular}
        \vspace{0.2cm}
    \subcaption{Smooth/Euclidean}
    \end{subtable} 
    \quad % for side by side tables
    \begin{subtable}[!h]{.5\textwidth}
        \centering
        \begin{tabular}{ p{2.9cm}p{2.7cm}c }
        & \textbf{CNS-PiloAstro} & \textbf{others} \\
        \textbf{Panc-AdenoCA} & 0.005 & - \\
        \textbf{Prost-AdenoCA} & 0.031 & - \\
        \textbf{all others} & - & - \\
        \end{tabular}
        \vspace{0.2cm}
    \subcaption{Smooth/Wasserstein}
    \end{subtable}  
    
\end{table}


