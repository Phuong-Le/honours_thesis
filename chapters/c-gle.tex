\chapter{Genomic Location Effect}\label{gle}

The tendency for mutations to occur in closed \gls{chromatin} regions has been reported both in cancer and other mutagenesis processes \citep{Polak2015,Prendergast2007ChromatinGenome}. This is hypothetically because closed chromatin regions, despite being less exposed to mutagens, are harder for repair systems to reach \citep{Prendergast2007ChromatinGenome,Teng1997ExcisionSequences, Morse2002PhotoreactivationCerevisiae}. Based on the premise that different cell-types harbour different chromatin structures \citep{Kundaje2015IntegrativeEpigenomes}, it is reasonable to expect that when developed into tumours, mutations are allocated differently between different cell-types. This chapter shows further evidence, through a G-test of independence and the \gls{or} statistic, that in cancers, mutations do tend to occur in closed rather than open chromatin regions. The degree at which this phenomenon occurs varies across cancer types. In addition, the chapter also uses the \gls{bootstrap} to confirm that \gls{gle} alone, without the chromatin status input, are significantly different in different cancers. 