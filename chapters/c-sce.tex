\chapter{Sequence Context Effect}\label{sce}

Different cancers develop under the influence of different mutagens, thus each cancer type possesses a distinctive mutation composition. Regarding individual mutation, each base change is closely integrated with the bases next to it \citep{Zhu2017,Zhu2020,Vinson2012CGMethylation}. In fact, previous work in the field has extracted signatures of numerous mutagens based on the 3-mer context \citep[see \gls{kmer};][]{Alexandrov2020,Alexandrov2013}. Notable among these is the distinctive signature called SBS4 of lung squamous-cell carcinoma, which, presumably due to tobacco smoke, consists more C[C$\rightarrow$A]A and C[C$\rightarrow$A]C than any other mutations. Using a measure of information richness, \gls{re}, this chapter demonstrates the importance of both the composition of base changes and their flanking bases in characterising the carcinogenesis pattern. Above all, the chapter shows that the \gls{sce} is not just predominated by immediate flanking bases (\textit{i.e.} 3-mers). Rather, there is certain value in factoring larger sequence contexts into analysing cancer mutation composition.

