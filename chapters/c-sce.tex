\chapter{Sequence Context Effect}\label{sce}

Different cancers develop under the influence of different conditions, especially mutagens and repair systems, thus each cancer type possesses a distinctive mutation composition. Regarding individual mutation, each base change is closely integrated with the bases next to it \citep{Zhu2017,Zhu2020,Vinson2012CGMethylation}. The question is whether flanking bases beyond 3-mers, specifically positions -2 \& +2, have an influence on whether the mutations occur. Using a measure of information richness, \gls{re}, this chapter demonstrates the importance of both the composition of base changes and their flanking bases in characterising the carcinogenesis pattern, with evidence of strand symmetry. Above all, the chapter shows that the \gls{sce} is not just contributed by immediate flanking bases (\textit{i.e.} 3-mers). Rather, there is certain value in factoring larger sequence contexts into analysing cancer mutation composition.

\section{Base substitutions are indicative of cancers}

\subsection{Base substitutions alone are a rich source of information}


\section{Information available in both inner and outer flanking bases to the mutations}