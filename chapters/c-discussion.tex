\chapter{Conclusion and Discussion}\label{discussion}

This project has shown that both mutation location in the genome and the composition of mutations were important characteristics of the cancer mutation profile. Regarding the genomic location effect (GLE), mutations were shown to preferably occur in closed chromatin regions over open chromatin regions. GLE helped characterise the mutation profile because it differed significantly between cancers. More importantly, the smooth representation was better at extracting information from GLE than the bin representation for cancer classification (Chapter \ref{gle}). Regarding the sequence context (SCE) for mutation composition, this project has shown that SCE contributes a considerable amount of information to the cancer mutation profile. Cancers had very different composition of mutations and the difference came from both the base substitutions and the flanking bases. Mutation composition was strand-symmetric and there was more information from transitions than there was from transversions (Chapter \ref{sce}). When training a distance-based KNN classifier, both GLE and SCE were shown to be predictive of cancers, SCE was the dominating predictor (Chapter \ref{ml}).

\section{Genomic location effect}
\subsection{Speculation of the mechanisms driving GLE}
My conclusion that mutations tend to occur in closed regions agree with previous observations \citep{Polak2015,Fujimoto2016Whole-genomeCancer,Prendergast2007ChromatinGenome}. Notably, \citet{Polak2015} used different measures of chromatin structure, including DHS and histone marks such as H3K4me1, to predict mutation density in Mbp bins. While the $OR$ mislabelling experiment of this project found that chromatin structure of the original cells was influential on GLE but unlikely to determine whether GLE differed between cancers (Section \ref{gle:mixed_or}), \citet{Polak2015} showed that chromatin structure data from correct cell types of origin were more correlated with the mutation density of the cancer than mislabelled cell types. The two results were not directly compatible, as the difference might have come from the different set of cancers investigated and the different types of data representing chromatin status (explained below). However, looking at the reported variance explained $R^2$ from \citet{Polak2015}, their claim was only really convincing for Skin-Melanoma and Liver-HCC, which are two distinctive cancers even in my project. 

Several factors could potentially interfere with the reliability of the DHS data recruited to represent chromatin status. First, for simplicity, I used the binary representation of DHS that treats genomic regions as either open or closed regions. However, the accessibility of DNA is actually more complicated, as there is variation in the openness of open chromatin regions \citep{Boyle2008High-ResolutionGenome}. A continuous representation instead of the binary representation used in this project would therefore be more accurate. Second, there was a degree of uncertainty in the identification of the cells of origin. This is particularly true for medulloblastoma, whose cell types of origin is still just a speculation \citep{Penas2015TheMedulloblastoma} but less true in skin melanoma, whose original cell type is quite established to be melanocyte \citep{Lin2007MelanocytePigmentation}. Besides, each DHS data for a cell type provided by ENCODE came from just one individual \citep{Thurman2012TheGenome}. Third, chromatin structure could be modified during carcinogenesis, for example due to mutations in chromatin modifier genes \citep{Makova2015TheGenome}. While this is an identified problem, there is no easy way to account for this given the availability of accessible data that I am aware of. I expect that using DHS data for the cancer itself rather than its original cell type is unlikely the solution because the modifications in chromatin structure differ between samples of similar cancer type across different stages of cancer.

Assuming DHS data is reliable, I hypothesise another explanation for the ability of GLE to predict cancers: the interaction between chromatin structure and the distribution of base compositions across the genome rather than the chromatin structure itself. This is explained in Figure \ref{}. Essentially, I hypothesise that some genomic regions have higher mutation density not only because they are in the closed chromatin but also because they contain bases that are prone to be targeted by a specific mutagen. Given that chromatin structure affects GLE, this explanation will make sense if the variation in chromatin structure is not great enough to shape the diversity of GLE. Indeed, \citet{Gilbert2004ChromatinFibers} showed that open chromatin regions contain gene clusters, hypothetically for expression and protection of genes \citep{Gilbert2004ChromatinFibers,Gazave2005DoesDamage}. This suggests a certain degree of conservativeness in chromatin structure as genes universal to all cell types will be more likely located in open chromatin regions. My proposed explanation can be evaluated by examining the association between the types of substitutions (whether the wildtype is A/T or C/G) and their location (whether they locate in closed or open chromatin regions) as well the association between the cancer type (whether mutations with wildtype A/T are from cancer A or B) and their genomic location with respect to the cancers (closed or open regions). 

\subsection{Representing GLE}
In this work, I showed that the smooth representation of GLE was better than the bin representation for both the bootstrap hypothesis testing (Section \ref{gle:bootstrap}) and the GLE-based classification (Section \ref{ml:gle}). For the bootstrap study, the p-value was not necessarily contrasted against a significance threshold, but it really was an indication of the ``extremity'' of the observed distance between cancers against the null. Interestingly, the Wasserstein distance somehow redeemed the performance of the bin representation compared to the smooth representation. As previously mentioned (Figure \ref{fig:mutdistribution_demo} of Section \ref{intro:gle}), the drawback of the bin representation is that it imposes arbitrary rigid boundaries to the genome. However, looking at its definition (Methods Figure \ref{fig:wasserstein_demo}), the Wasserstein allows comparisons of bins with different coordinates, thereby ``breaking'' the rigid boundaries. In other words, the Wasserstein distance itself has a smoothing effect. Regarding the 1 Mbp bin size, \citet{Hodgkinson2012TheGenomes} stated that the variation in GLE was detected at the 1 Mbp scale, but blurred out at 10 Mbp scale. Although the two scales are quite different, and the Mbp scale was not demonstrated to be the lower limit, this established that the use of 1 Mbp was reasonable. Accordingly, the choice of using the smooth representation or the bin representation combined with Wasserstein distance should therefore depend on computing performance. 

\section{Sequence context effect}
\subsection{Patterns of mutation composition}
Using the measure of information $RE$, I detected patterns of mutation composition that were compatible with previous work \citep[Chapter \ref{sce};][]{Alexandrov2020}. In particular, \citet{Alexandrov2020} decomposed the composition of mutations using non-negative matrix factorisation to identify the so-called ``mutation signatures''. They established that signature SBS7a, presumably associated with UV exposure, was found in all skin melanoma samples and was only found in skin melanoma. This signature is predominated by the C$\rightarrow$T substitution, especially when there is a T at flanking position -1. This is consistent with my observation in Figures \ref{fig:spectra_skin} and \ref{fig:transitions_skin}. Likewise, signature SBS12 detected only in hepatocellular carcinoma was enriched in the T$\rightarrow$C substitution, which is in line with the observation in Figure \ref{fig:spectra_liver}. The work of \citet{Alexandrov2020} is a detailed decomposition of mutation compositions, where each cancer can have multiple signatures. In the meantime, my work, developed on top of \citet{Zhu2017}, was analogical to a general summary of mutation composition. 

\subsection{Strand symmetry}
Both $RE$ (Chapter \ref{sce}) and the SCE-based classifier (Section \ref{ml:sce}) of my project showed evidence of strand symmetry, in the sense that reverse complementary mutations are counted as the same category (introduced in Figure \ref{fig:motif_symmetric_demo} of Section \ref{intro:sce}). This representation, which I referred to as semi-symmetry, was adopted by most publication that I am aware of \citep{Alexandrov2020,Jiao2020,Zhang2020}. However, \citet{Zhu2017} showed that the compositions of flanking bases to a substitution significantly differed between two DNA strands in skin melanoma. This is not necessarily contradictory with my finding, but it reflects the fact that the data I used did not specify what mutations occurred on which DNA strand. Nevertheless, this shows that given this type of data, which is often the case, the semi-symmetry representation should be sufficient to capture information.

\subsection{Flanking bases beyond 3-mers}
There was inconsistency in the $RE$ of flanking positions (Section \ref{sce:nbr}) and the SCE-based classifier (Section \ref{ml:sce}) with respect to the importance of bases at positions -2 and +2. Even though information was observed in these positions using $RE$, incorporating them into the classifier (5-mer) led to a drop in accuracy. I attributed this to the size of the vector representing 5-mers. Both the attempt to break the 5-mer vector down into shorter vectors and the introduction of semi-symmetry improved accuracy for 5-mers, which supports my speculation. It is worth noting that the fully symmetric representation decreased accuracy instead of improving it. This representation was an attempt to shorten the SCE vector without a biological basis; therefore, its results indicate that the length of the representation vector was not a determinant of accuracy without a good biological rationale. In comparison with the literature, \citet{Zhu2020} showed that it was possible to identify the origin of a single mutation based on its flanking bases. \citep{Zhang2020} showed that using the sequence context of the 9-mer context was the best at predicting cancers, both for simulated data and for breast cancer subtypes. Their results were obtained using the support vector machines, which is another distance based classifier algorithm. This is a great support for the potential of incorporating bases beyond 3-mers. However, while \citet{Zhang2020} also split up large context sizes, their representation forced bases of the same distance to the substitution (\textit{e.g. positions -1 and +1}) to have the same weight. As previously seen from Figure \ref{sce:nbr}, for a mutation (\textit{e.g.} C$\rightarrow$T), the contribution of position +1 to the information context was different from that of position -1.

\section{Misclassification and improvement of accuracy}
\section{Application in cancer diagnosis}
\section{Conclusion and future directions}