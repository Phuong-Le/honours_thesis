\chapter{Conclusion and Discussion}\label{discussion}

This project has shown that both mutation location in the genome and the composition of mutations were important characteristics of the cancer mutation profile. Regarding the genomic location effect (GLE), mutations were shown to preferably occur in closed chromatin regions over open chromatin regions. GLE helped characterise the mutation profile because it differed significantly between cancers. More importantly, the smooth representation was better at extracting information from GLE than the bin representation for cancer classification (Chapter \ref{gle}). Regarding the sequence context (SCE) for mutation composition, this project has shown that SCE contributes a considerable amount of information to the cancer mutation profile. Cancers had very different composition of mutations and the difference came from both the base substitutions and the flanking bases. Mutation composition was strand-symmetric and there was more information from transitions than there was from transversions (Chapter \ref{sce}). When training a distance-based KNN classifier, both GLE and SCE were shown to be predictive of cancers, SCE was the dominating predictor (Chapter \ref{ml}).

\section{Genomic location effect}
\subsection{Speculation of the mechanisms driving GLE}
My conclusion that mutations tend to occur in closed regions agree with previous observations \citep{Polak2015,Fujimoto2016Whole-genomeCancer,Prendergast2007ChromatinGenome}. Notably, \citet{Polak2015} used different measures of chromatin structure, including DHS and histone marks such as H3K4me1, to predict mutation density in Mbp bins. While the $OR$ mislabelling experiment of this project found that chromatin structure of the original cells was influential on GLE but unlikely to determine whether GLE differed between cancers (Section \ref{gle:mixed_or}), \citet{Polak2015} showed that chromatin structure data from correct cell types of origin were more correlated with the mutation density of the cancer than mislabelled cell types. The two results were not directly compatible, as the difference might have come from the different set of cancers investigated and the different types of data representing chromatin status (explained below). However, looking at the reported variance explained $R^2$ from \citet{Polak2015}, their claim was only really convincing for Skin-Melanoma and Liver-HCC, which are two distinctive cancers even in my project. 

Several factors could potentially interfere with the reliability of the DHS data recruited to represent chromatin status. First, for simplicity, I used the binary representation of DHS that treats genomic regions as either open or closed regions. However, the accessibility of DNA is actually more complicated, as there is variation in the openness of open chromatin regions \citep{Boyle2008High-ResolutionGenome}. A continuous representation instead of the binary representation used in this project would therefore be more accurate. Second, there was a degree of uncertainty in the identification of the cells of origin. This is particularly true for medulloblastoma, whose cell types of origin is still just a speculation \citep{Penas2015TheMedulloblastoma} but less true in skin melanoma, whose original cell type is quite established to be melanocyte \citep{Lin2007MelanocytePigmentation}. Besides, each DHS data for a cell type provided by ENCODE came from just one individual \citep{Thurman2012TheGenome}. Third, chromatin structure could be modified during carcinogenesis, for example due to mutations in chromatin modifier genes \citep{Makova2015TheGenome}. While this is an identified problem, there is no easy way to account for this given the availability of accessible data that I am aware of. I expect that using DHS data for the cancer itself rather than its original cell type is unlikely the solution because the modifications in chromatin structure differ between samples of similar cancer type across different stages of cancer.

Assuming DHS data is reliable, I hypothesise another explanation for the ability of GLE to predict cancers: the interaction between chromatin structure and the distribution of base compositions across the genome rather than the chromatin structure itself. This is explained in Figure \ref{}. Essentially, I hypothesise that some genomic regions have higher mutation density not only because they are in the closed chromatin but also because they contain bases that are prone to be targeted by a specific mutagen. Given that chromatin structure affects GLE, this explanation will make sense if the variation in chromatin structure is not great enough to shape the diversity of GLE. Indeed, \citet{Gilbert2004ChromatinFibers} showed that open chromatin regions contain gene clusters, hypothetically for expression and protection of genes \citep{Gilbert2004ChromatinFibers,Gazave2005DoesDamage}. This suggests a certain degree of conservativeness in chromatin structure as genes universal to all cell types will be more likely located in open chromatin regions. My proposed explanation can be evaluated by examining the association between the types of substitutions (whether the wildtype is A/T or C/G) and their location (whether they locate in closed or open chromatin regions) as well the association between the cancer type (whether mutations with wildtype A/T are from cancer A or B) and their genomic location with respect to the cancers (closed or open regions). 

\subsection{Representing GLE}


\section{Sequence context effect}
\section{Misclassification and improvement of classification accuracy}
\section{Application in cancer diagnosis}
\section{Conclusion and future directions}