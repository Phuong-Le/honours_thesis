\chapter{Classification}\label{ml}

Chapter \ref{gle} demonstrates that the \gls{gle} contributes to characterising the cancer mutagenesis patterns. Chapter \ref{sce} shows evidence that there is an informational advantage in exploiting the bases outside 3-mers surrounding the changed bases. Nonetheless, to manipulate the information from \gls{gle} and \gls{sce}, they have to be represented correctly. A sensible instinct is that a more suitable representation of a feature gives higher accuracy than a less suitable representation. This chapter acts as both an application and a validation for chapters \ref{gle} and \ref{sce}. In particular, section \ref{ml:gle} shows that the smooth representation of \ref{gle} is a better representation than binning the genome. Section \ref{ml:sce} shows that \textcolor{blue}{blah blah}. Additionally, section \ref{ml:both} demonstrates how to combine two factors with different units, like GLE and SCE, in a joint model in an attempt to further improve accuracy and to weigh the importance of each in the presence of the other.

While each of the next sections works on different inputs, all sections follow the same procedure for training the classifier (\ref{methods:ml_workflow}). Briefly, a small proportion of data is set aside as a test set, which is then used to evaluate model performance.

\section{Classifier based on GLE}\label{ml:gle}
\section{Classifier based on SCE}\label{ml:sce}
\section{Classifier based on the combination of GLE and SCE}\label{ml:both}