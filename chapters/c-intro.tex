\chapter{Introduction}\label{intro}

From a genetics point of view, the process of \gls{carcinogenesis} is operated and characterised by mutations \citep{Stratton2009}. While every cancer has a different set of mutations, certain mutation patterns have been found exclusive to several cancer types \citep{Alexandrov2013,Polak2015,Campbell2020}. Since all cancers originate from a normal cell \citep{Hanahan2011HallmarksGeneration}, the patterns of cancer \gls{mutagenesis} likely reflect the mutation tendency in the original cell type to some extent. During its development, the phenotype of a cancer sample could diverge from that of the original cell, but virtually all mutations prior to the divergence remain in the cancer genome. These divergence events can be caused by driver mutations, which promote cancer cell proliferation \citep{Pon2015}. An average cancer sample consists of only about 4-5 drivers, the rest are passenger mutations, which have neutral effects on the cancer progress \citep{Campbell2020}. Together, the whole history of mutations in a cancer sample helps characterise its mutation profile (Figure \ref{}). This project seeks to study the invaluable information extracted from the cancer mutation profiles, which could potentially help guiding treatment. Furthermore, the project seeks to exploit this information to predict cancer types by training a \gls{classifier} that only relies on genomic sequencing data. In the long term, such a classifier could be an additional diagnostic tool to existing clinical approaches such as cytology or biopsy \citep{Stone1995Biopsy:Pitfalls}. In this era of next generation sequencing, liquid biopsies are gaining interest as a powerful non-invasive method for early cancer diagnosis because they involve screening for circulating tumour DNA in the blood rather than obtaining samples from a suspected local tissue \citep{Chen2019Next-generationDetection}. Developing a genome-based classifier model means that liquid biopsies could inform not only whether a cancer is present, but also where and what cancer is occurring at an early stage.  

Regarding the scope of the project, two aspects of of a cancer mutation profile were studied: where mutations tend to be found in the genome (hereafter the Genomic Location Effect, GLE) and what \gls{base} changes tend to be found in which genomic sequence context (hereafter the Sequence Context Effect, SCE). Secondly, the project is limited to point mutations, which are the most abundant type of mutations in cancer \citep{Alexandrov2020}. Thirdly, due to the different mutation rates between drivers and passengers, they should be considered separately - this project focuses on passenger mutations \citep{McFarland2014Tug-of-warProcesses}. Finally, this project only investigates \glspl{sommut} as opposed to \glspl{germline_mut}, acknowledging that germline mutations could themselves be a risk factor of cancer. The reason for this is that germline mutations reflect a process that definitely occurs prior to carcinogenesis. Strictly speaking, somatic mutations can also occur before carcinogenesis, but they still arguably reflect the environment in which cancer develops. 

To outline this chapter, section \ref{intro:gle} and \ref{intro:sce} review what is known about GLE and SCE, respectively. Section \ref{intro:ml} then briefly introduces the statistical approaches used to study GLE and SCE, as well as how they can be used to train a machine learning classifier.

\section{Genomic location effect (GLE)}
\label{intro:gle}
\section{Sequence context effect (SCE)}
\label{intro:sce}
\section{Computational techniques for studying GLE and SCE}
\label{intro:ml}