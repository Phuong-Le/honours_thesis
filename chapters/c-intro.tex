\chapter{Introduction}\label{intro}

From a genetics point of view, the process of \gls{carcinogenesis} is operated and characterised by mutations \citep{Stratton2009}. While every cancer has a different set of mutations, certain mutation patterns have been found exclusive to several cancer types \citep{Alexandrov2013,Polak2015,Campbell2020}. Since all cancers originate from a normal cell \citep{Hanahan2011HallmarksGeneration}, it is likely that the patterns of cancer \gls{mutagenesis}, to some extent, reflect the mutation tendency in the original cell type. During its development, the phenotype of a cancer sample could diverge from that of the original cell, but virtually all mutations prior to the divergence remain in the cancer genome. These divergence events can be caused by driver mutations, which promote cancer cell proliferation \citep{Pon2015}. An average cancer sample consists of only about 4-5 drivers, the rest are passenger mutations, which have neutral effects on the cancer progress \citep{Campbell2020}. Together, the whole history of mutations in a cancer sample helps characterise its mutation profile (Figure \ref{fig:drivers_demo}). This project seeks to study the invaluable information extracted from the cancer mutation profiles, which could potentially help guiding treatment. Furthermore, the project seeks to exploit this information to predict cancer types by training a \gls{classifier} that only relies on genomic sequencing data. In the long term, such a classifier could be an additional diagnostic tool to existing clinical approaches such as cytology or biopsy \citep{Stone1995Biopsy:Pitfalls}. In this era of next generation sequencing, liquid biopsies are gaining interest as a powerful non-invasive method for early cancer diagnosis because they involve screening for circulating tumour DNA in the blood rather than obtaining samples from a suspected local tissue \citep{Chen2019Next-generationDetection}. Developing a genome-based classifier model means that liquid biopsies could inform not only whether a cancer is present, but also where and what cancer is occurring at an early stage.  

Regarding the scope of the project, two aspects of of a cancer mutation profile were studied: where mutations tend to be found in the genome (hereafter the Genomic Location Effect, GLE) and what \gls{base} changes tend to be found in which genomic sequence context (hereafter the Sequence Context Effect, SCE). Secondly, the project is limited to point mutations, which are the most abundant type of mutations in cancer \citep{Alexandrov2020}. Thirdly, due to the different mutation rates between drivers and passengers, they should be considered separately - this project focuses on passenger mutations \citep{McFarland2014Tug-of-warProcesses}. Finally, this project only investigates \glspl{sommut} as opposed to \glspl{germline_mut}, acknowledging that germline mutations could themselves be a risk factor of cancer. The reason for this is that germline mutations reflect a process that definitely occurs prior to carcinogenesis. Strictly speaking, somatic mutations can also occur before carcinogenesis, but they still arguably reflect the environment in which cancer develops (Figure \ref{fig:drivers_demo}). 

\begin{figure}[ht!]
    \centering
    \includegraphics[scale=0.78]{graphics/drivers_demo.pdf}
    \caption{\textbf{Timeline of a carcinogenesis process.} Mutations are almost always retained in the genome after each stage of carcinogenesis (\#muts means the number of mutations). Together, all mutations in a cancer cell make up its mutation profile. For the purpose of this project, only passenger somatic point mutations were considered.}
    \label{fig:drivers_demo}
\end{figure}

% \textbf{Timeline of a carcinogenesis process.} Mutations are generally retained in the genome after each stage of carcinogenesis, even though carcinogenesis and driver events could change the phenotype of the cells, (\#muts means the number of mutations). Together, all mutations available in a cancer cell make up its mutation profile. For the purpose of this project, only somatic mutations are considered because germline mutations are not the product of the environment of the differentiated cells in which cancer develops.

To outline this chapter, section \ref{intro:gle} and \ref{intro:sce} review what is known about GLE and SCE, respectively. Section \ref{intro:ml} then briefly introduces the statistical approaches used to study GLE and SCE, as well as how they can be used to train a machine learning classifier.

\section{Genomic location effect (GLE)}
\label{intro:gle}
Certain regions of the genome are more prone to mutations than others. These regions are distributed at different locations of the genome in different cell-types \citep{Polak2015}. This is because different cells have different chromatin structures \citep{Abascal2020ExpandedGenomes}, which are determined by how hidden DNA is in the chromatin complex across the genome. The accessibility of DNA to factors such as transcription, mutagens and repair systems can be measured by Dnase I hypersensitivity \citep[DHS;][]{Liu2019AApplications}. Highly accessible DNA means open chromatin status, and vice versa. Because mutation is the product of both \glspl{mutagen} and repairs, it is reasonable to expect that the most vulnerable regions are those with strong mutagens and weak repairs. In reality, DNA in the closed chromatin region is less exposed to both mutagens and repairs \citep[Figure \ref{fig:chromatin_demo};][]{Teng1997ExcisionSequences, Morse2002PhotoreactivationCerevisiae}. However, it has been reported that closed chromatin regions are more likely to harbour mutations than open regions \citep{Polak2015,Prendergast2007ChromatinGenome}, suggesting that the repair effect is generally stronger. The diversity of chromatin structures and of how mutagen/repair system works in different cells makes genomic location of mutations (hereafter GLE) an important characteristic of the cancer mutation profile. Chapter \ref{gle} of this project seeks to formally test how GLE differs in closed \textit{v.s.}open chromatin regions for each cancer. Chapter \ref{gle} also evaluates the difference in GLE between cancers, without prior knowledge of chromatin status. This serves to assess the information content available in GLE so it can be exploited to predict cancer types in Chapter \ref{ml}.

\begin{figure}[ht!]
    \centering
    \includegraphics[scale=0.24]{graphics/chromatin_demo.png}
    \caption{\textbf{The distribution of mutations across the genome is hypothesised to be influenced by cell chromatin structure.} DNA in closed chromatin regions is less accessible to mutagens and DNA repair, and is more prone to mutations. Different cell types have different chromatin structures as well as different repair systems, making the genomic location of mutations a potential source for distinguishing cancers.  TF means transcription factor, Pol II means polymerase II. Figure modified from \citet{Klemm2019ChromatinEpigenome}.}
    \label{fig:chromatin_demo}
\end{figure}


\section{Sequence context effect (SCE)}
\label{intro:sce}
\section{Computational approaches for exploiting GLE and SCE to predict cancers}
\label{intro:ml}