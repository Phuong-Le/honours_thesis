\newpage
\setcounter{table}{0}
\setcounter{figure}{0}
\setcounter{section}{0}
\renewcommand\thesection{A.\arabic{section}} 
\renewcommand\thefigure{A\arabic{figure}} 
\renewcommand\thetable{A\arabic{table}} 

\chapter*{Appendices}
\addcontentsline{toc}{chapter}{Appendices}

\section{Cancers of interest and putative original cells}
\begin{table}[hp!]
\centering
\caption{\textbf{12 cancers of interest, their abbreviation and their putative original cell types.} The \textbf{Cell line code} column is the code for DHS data of the original cell type, set by the ENCODE project; \textbf{Source} cites the publication that establishes the relationship between the cancer and the original tissue.}
\label{tab:encode}
\begin{tabulary}{\textwidth}{ LlLLR }
\toprule
\bf{Cancer Type} & \bf{Abbreviation} & \bf{Tissue of origin} & \textbf{Cell line code} & \bf{Source} \\
\toprule
Osteosarcoma & Bone-Osteosarc & Connective cells of the bone & Osteobl & \citet{Alfranca2015BoneDevelopment} \\ \hline

Breast Adenocarcinoma & Breast-AdenoCa & Breast epithelium & HMEC & \citet{Boyce2007BreastCancer} \\ \hline

Medulloblastoma &  CNS-Medullo &  Cerebellum & Cerebellum\_OC & \citet{Penas2015TheMedulloblastoma} \\ \hline

Pilocytic Astrocytoma & CNS-PiloAstro & Astrocytes & HAc & \citet{Collins2015PilocyticMarkers}\\ \hline

Kidney Rectal Cell Carcinoma & Kidney-RCC & Renal proximal tubule epithelium & RPTEC & \citet{Hsieh2017RenalCarcinoma} \\ \hline

Liver Hepatocellular Carcinoma & Liver-HCC & Hepatocytes & Hepatocytes & \citet{Gissen2015StructuralDisease} \\ \hline

B Cell Non-Hodgekin Lymphoma & Lymph-BNHL & B cells & B cells CD20+ RO01794 & \citet{Shankland2012Non-HodgkinLymphoma} \\ \hline

Chronic Lymphocytic Leukemia & Lymph-CLL & B cells & B cells CD20+ RO01794 & \citet{Hallek2018ChronicLeukaemia} \\ \hline

Pancreatic Andenocarcinoma & Panc-AdenoCa & Exocrine cells & HPDE6-E6E7 & \citet{Vareedayah2018PancreaticAdenocarcinoma} \\ \hline

Pancreatic Endocrine Cancer & Panc-Endocrine & Islet cells & PanIslets & \citet{Nakakura2007IsletRegion} \\ \hline

Prostate Adenocarcinoma & Prost-AdenoCa & Prostate epithelium & RWPE1 & \citet{Lee2011OverviewPathology} \\ \hline

Skin Melanoma & Skin-Melanoma & Melanocytes &  Melano & \cite{Lin2007MelanocytePigmentation} \\
\bottomrule

\end{tabulary}
\end{table}

% . DHS data for these cells is downloaded from either \href{https://genome.ucsc.edu/cgi-bin/hgFileUi?db=hg19&g=wgEncodeOpenChromDnase}{Duke} or \href{https://genome.ucsc.edu/cgi-bin/hgFileUi?db=hg19&g=wgEncodeUwDnase}{UW} project

\section{PCAWG mutation summary}
\vspace{1cm}
\begin{table}[hp!]
\centering
\caption{\textbf{Mutation summary for the 12 cancers of interest}. These are the somatic mutations from the ICGC portion of the PCAWG project.}
\label{tab:mutation_summary}
\begin{tabulary}{\textwidth}{ lRRRR }
\hline
\bf{Cancer Type} & \bf{Number of donors} & \bf{Average number of mutations by donors} & \bf{Total number of mutations} & \bf{Standard deviation of mutations by donors} \\
\hline
  Bone-Osteosarc &               44 &                   3792 &                    166845 &                       3003.40 \\
  Breast-AdenoCa &              113 &                   6318 &                    713855 &                       8610.10 \\
     CNS-Medullo &              146 &                   1438 &                    209997 &                       1053.20 \\
   CNS-PiloAstro &               89 &                    247 &                     22020 &                        220.70 \\
      Kidney-RCC &               74 &                   7188 &                    531886 &                       5774.70 \\
       Liver-HCC &              264 &                  12582 &                   3321521 &                       6731.50 \\
      Lymph-BNHL &               98 &                  11478 &                   1124881 &                      14534.10 \\
       Lymph-CLL &               95 &                   2381 &                    226242 &                        889.60 \\
    Panc-AdenoCA &              239 &                   7012 &                   1675781 &                       8041.50 \\
  Panc-Endocrine &               85 &                   3042 &                    258564 &                       3284.80 \\
   Prost-AdenoCA &              191 &                   5238 &                   1000496 &                       8892.70 \\
   Skin-Melanoma &               70 &                 111014 &                   7770980 &                     145095.00 \\
\hline
\end{tabulary}
\end{table}

\vspace{2cm}
\begin{figure}[h!]
    \centering
    \includegraphics[scale=0.66]{graphics/mutation_summary.pdf}
    \caption{\textbf{Number of mutations (log scale) for each cancer.} Each dot represents a donor. Most patients have approximately 1000-10000 mutations, but there is a great variation.}
    \label{fig:mutation_summary}
\end{figure}


\newpage
\section{How cancer original cell types are related by DHS}
% \begin{figure}[h!]
%     \begin{subfigure}{.5\textwidth}
%     \centering
%     \includegraphics[scale=0.7]{graphics/encode_pca_1_2.pdf}
%     \caption{PC2 \textit{v.s.} PC1}
%     \end{subfigure}
%     ~
%     \begin{subfigure}{.5\textwidth}
%     \centering
%     \includegraphics[scale=0.7]{graphics/encode_pca_1_3.pdf}
%     \caption{PC3 \textit{v.s.} PC1}
%     \end{subfigure} \\
%     \caption{\textbf{PCA}.}
%     \label{fig:encode_pca}
% \end{figure}

\begin{figure}[ht!]
  \begin{minipage}[c]{\textwidth}
    \begin{subfigure}{.5\textwidth}
    \centering
    \includegraphics[scale=0.7]{graphics/encode_pca_1_2.pdf}
    \caption{PC2 \textit{v.s.} PC1}
    \end{subfigure}
    ~
    \begin{subfigure}{.5\textwidth}
    \centering
    \includegraphics[scale=0.7]{graphics/encode_pca_1_3.pdf}
    \caption{PC3 \textit{v.s.} PC1}
    \end{subfigure} \\
  \end{minipage}\hfill
  \vspace{1cm}
  
  \begin{minipage}[c]{\textwidth}
    \centering
    \begin{tabulary}{\textwidth}{ ll }
    \toprule
    \textbf{Original cell abbreviation} & \bf{Cancer Type}  \\
    \toprule
    Osteoblast & Osteosarcoma \\
    
    BreastEpi & Breast-AdenoCa \\
    
    Cerebellum &  CNS-Medullo  \\
    
    CereberumAstro & CNS-PiloAstro \\
    
    KidneyRPTEC & Kidney-RCC \\
    
    Hepatocyte & Liver-HCC \\
    
    BCell & Lymph-BNHL, Lymph-CLL \\
    
    PancDuct & Panc-AdenoCa \\
    
    IsletCell & Panc-Endocrine \\
    
    ProstEpi & Prost-AdenoCa \\
    
    Melanocyte & Skin-Melanoma \\
    \bottomrule
    
    \end{tabulary}
    
    % . DHS data for these cells is downloaded from either \href{https://genome.ucsc.edu/cgi-bin/hgFileUi?db=hg19&g=wgEncodeOpenChromDnase}{Duke} or \href{https://genome.ucsc.edu/cgi-bin/hgFileUi?db=hg19&g=wgEncodeUwDnase}{UW} project
  \end{minipage}\hfill
  \vspace{0.9cm}
  
  \begin{minipage}[c]{\textwidth}
    \caption{
      \textbf{Some cancers were more related in terms of chromatin structures than others.} Here, I visualised the relative coordinates of the original cell types for cancers on the most informative dimensions (principle coordinates, PC). The informativeness of each PC manifests in the percentage of variance explained. This was done by multidimensional scaling of the pairwise distance between cell types. The distance between two cell types was computed based on the intersection between their open chromatin regions. 
    } \label{fig:encode_pca}
  \end{minipage}
\end{figure}

\vspace{1cm}
\begin{table}[hp!]
\centering
\caption{\textbf{Abbreviation of original cell types in Figure \ref{fig:encode_pca} and the corresponding cancers}}
\label{tab:encode_pca}
\begin{tabulary}{\textwidth}{ ll }
\toprule
\textbf{Original cell type} & \bf{Cancer Type}  \\
\toprule
Osteoblast & Osteosarcoma \\

BreastEpi & Breast-AdenoCa \\

Cerebellum &  CNS-Medullo  \\

CereberumAstro & CNS-PiloAstro \\

KidneyRPTEC & Kidney-RCC \\

Hepatocyte & Liver-HCC \\

BCell & Lymph-BNHL, Lymph-CLL \\

PancDuct & Panc-AdenoCa \\

IsletCell & Panc-Endocrine \\

ProstEpi & Prost-AdenoCa \\

Melanocyte & Skin-Melanoma \\
\bottomrule

\end{tabulary}
\end{table}

% . DHS data for these cells is downloaded from either \href{https://genome.ucsc.edu/cgi-bin/hgFileUi?db=hg19&g=wgEncodeOpenChromDnase}{Duke} or \href{https://genome.ucsc.edu/cgi-bin/hgFileUi?db=hg19&g=wgEncodeUwDnase}{UW} project

\newpage
\section{Supplementary equations for smoothing GLE}
\subsection{Gaussian kernel density estimation}

Kernel functions estimate the density of data at a particular location by measuring the distance from that location to all observed data points. To estimate mutation density across the genome, I used the Gaussian kernel, as follows:

\begin{equation}
    K(u) = \frac{1}{2\pi} e^{\frac{-u^2}{2}}
    \label{eq:gaussian}
\end{equation}

where $K$ is the kernel function, $u$ is a variable - \textit{i.e.} $\frac{x-X{i}}{h}$ in equation \ref{eq:density}

\subsection{Bandwidth choice}
The bandwidth $h$ determines the level of smoothing, or the distance at which the an observed mutation can meaningfully influence the density $\hat{f}$ at location $x$, I used the Scott's method, as follows:

\begin{equation}
    h = 1.059 A n^{-1/5}
    \label{eq:bandwidth}
\end{equation}

where $h$ is the bandwidth, $A$ is a measure of spread of the observed mutation locations (\textit{i.e.} the smaller of (1) the standard deviation and the (2) inter-quartile range) and $n$ is the total number of mutations.

