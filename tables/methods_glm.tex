\vspace{0.2cm}
\begin{figure}[h!]
  \begin{minipage}[c]{0.5\textwidth}
    \captionof{table}{\textbf{Example of how GLM and RE can be used for categorical data in a contingency table}. In this example, the indicator (categorical) variable for cancer type $\lambda_{cancer}$ takes values cancer A or cancer B; the variable for the composition of base substitution $\lambda_{sub}$ takes values T$\rightarrow$A, T$\rightarrow$C and T$\rightarrow$G. To establish whether $\lambda_{cancer}$ is explanatory of $\lambda_{sub}$, we can perform a hypothesis test based on a null and a saturated model (formulae \ref{eq:spectra_demo}; the $counts$ corresponds to the $count$ term in equation \ref{eq:spectra_demo}). The hypothesis test outputs a p-value; if p-value is small enough, it can the explanatory relationship. The test also allows estimating $RE$, which measures how much information is contributed by each entry in the table.} 
    \label{tab:glm_demo}
  \end{minipage}\hfill
  \begin{minipage}[c]{0.48\textwidth}
    \begin{tabulary}{\columnwidth}{rRRR}
    \toprule
        & \textbf{T$\rightarrow$A} & \textbf{T$\rightarrow$C} & \textbf{T$\rightarrow$G}  \\
    \hline
        \textbf{Cancer A} & $count_{T>A}$ & $count_{T>C}$ & $count_{T>G}$  \\
        \textbf{Cancer B} & $count_{T>A}$ & $count_{T>C}$ & $count_{T>G}$  \\
    \bottomrule
    \end{tabulary}
    \vspace{1cm}
    \begin{equation}
        \begin{aligned}
            H_o: \ln{count} =& \lambda_{cancer} + \lambda_{sub}  & \\
            H_a: \ln{count} =& \lambda_{cancer} + \lambda_{sub} + \lambda_{cancer:sub}
        \end{aligned}
        \label{eq:spectra_demo}
    \end{equation}
  \end{minipage}
\end{figure}
