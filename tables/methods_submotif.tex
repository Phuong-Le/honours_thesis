\vspace{0.2cm}
\begin{table}[hp!]
\centering
\caption{\textbf{Summary of the properties of the representations that dissect 5-mer into smaller submotifs.} 2-submotif has 4 component vectors; 3-motif has 3 component vectors. \textbf{\#elements} are the number of elements available in each component vector. sub is the substitution}
\label{tab:submotif}
\begin{tabulary}{\textwidth}{ lllllL }
\toprule
 & \textbf{vector 1} & \textbf{vector 2} & \textbf{vector 3} & \textbf{vector 4} & \textbf{\#elements} \\
\hline
2-submotif & (sub, pos-2) & (sub, pos-1) & (sub, pos+1) & (sub, pos+2) & 48 \\
3-submotif & (sub, pos-2,-1) & (sub, pos-1,+1) & (sub, pos+1,+2) & & 192 \\
whole-5mer & (sub, all 4 pos) & & & & 3072 \\
\bottomrule
\end{tabulary}
\end{table}
