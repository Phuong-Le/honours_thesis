\begin{table}[hp!]
\centering
\caption{\textbf{12 cancers of interest, their abbreviation and their putative original cell types.} The \textbf{Cell line code} column is the code for DHS data of the original cell type, set by the ENCODE project; \textbf{Source} cites the publication that establishes the relationship between the cancer and the original tissue.}
\label{tab:encode_pca}
\begin{tabulary}{\textwidth}{ ll }
\toprule
\textbf{Original cell type} & \bf{Cancer Type}  \\
\toprule
Osteoblast & Osteosarcoma \\

BreastEpi & Breast-AdenoCa \\

Cerebellum &  CNS-Medullo  \\

CereberumAstro & CNS-PiloAstro \\

KidneyREPTEC & Kidney-RCC \\

Hepatocyte & Liver-HCC \\

BCell & Lymph-BNHL \\

BCell & Lymph-CLL \\

PancEpi & Panc-AdenoCa \\

IsletCell & Panc-Endocrine \\

ProstEpi & Prost-AdenoCa \\

Melanocyte & Skin-Melanoma \\
\bottomrule

\end{tabulary}
\end{table}

% . DHS data for these cells is downloaded from either \href{https://genome.ucsc.edu/cgi-bin/hgFileUi?db=hg19&g=wgEncodeOpenChromDnase}{Duke} or \href{https://genome.ucsc.edu/cgi-bin/hgFileUi?db=hg19&g=wgEncodeUwDnase}{UW} project