\vspace{0.2cm}
\begin{figure}[h!]
  \begin{minipage}[c]{0.4\textwidth}
    \captionof{table}{\textbf{Contingency table to examine the composition of base substitutions in Cancer A}. Cancer A is compared to a null cancer where all counts are the same. The resulting $RE$'s represent the excess/deficit of certain substitutions. This particular table computes the $RE$ set for substitutions whose wildtype is T. $\lambda_{cancer}$ is a categorical variable taking values \textbf{Cancer A} or \textbf{Null}. $\lambda_{sub}$ indicates whether the counted substitution is \textbf{T$\rightarrow$A}, \textbf{T$\rightarrow$C} or \textbf{T$\rightarrow$G}. For each cancer, three similar tables are required for substitutions of A, C and G.} 
    \label{tab:glm_spectra}
  \end{minipage}\hfill
  \begin{minipage}[c]{0.59\textwidth}
    \begin{tabulary}{\columnwidth}{lCCCR}
    \toprule
        & \textbf{T$\rightarrow$A} & \textbf{T$\rightarrow$C} & \textbf{T$\rightarrow$G}  & \textbf{total} \\
    \hline
        \textbf{Cancer A} & $count_{T>A}$ & $count_{T>C}$ & $count_{T>G}$ & $count_{tot}$ \\
        \textbf{Null} & $count_{tot}/3$ & $count_{tot}/3$ & $count_{tot}/3$ & $count_{tot}$ \\
    \bottomrule
    \end{tabulary}
    \vspace{0.9cm}
    \begin{equation}
        \begin{aligned}
            H_o: \ln{count} =& \lambda_{cancer} + \lambda_{sub}  & \\
            H_a: \ln{count} =& \lambda_{cancer} + \lambda_{sub} + \lambda_{cancer:sub}
        \end{aligned}
        \label{eq:spectra}
    \end{equation}
  \end{minipage}
\end{figure}
